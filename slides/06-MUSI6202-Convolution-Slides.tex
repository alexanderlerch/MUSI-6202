% move all configuration stuff into one file so we can focus on the content
\input{../shared/common}
\input{../shared/beamersetup}
%%%%%%%%%%%%%%%%%%%%%%%%%%%%%%%%%%%%%%%%%%%%%%%%%%%%%%%%%%%%%%%%%%%%%%%%%%%%%%%%%%
%%%%%%%%%%%%%%%%%%%%%%%%%%%%%%%%%%%%%%%%%%%%%%%%%%%%%%%%%%%%%%%%%%%%%%%%%%%%%%%%%%
% title information
\title[]{MUSI6202: Digital Signal Processing for Music}   
\author[alexander lerch]{alexander lerch} 
%\institute{~}
%\date[Alexander Lerch]{}
%\titlegraphic{\vspace{-16mm}\includegraphics[width=\textwidth,height=3cm]{title}}

\input{../shared/definitions}



\subtitle{Part 6: LTI Systems \& Convolution}

%%%%%%%%%%%%%%%%%%%%%%%%%%%%%%%%%%%%%%%%%%%%%%%%%%%%%%%%%%%%%%%%%%%%%%%%%%%%
\begin{document}
    % generate title page
	\title[]{Digital Signal Processing for Music}   
\author[alexander lerch]{alexander lerch} 
%\institute{~}
%\date[Alexander Lerch]{}
\titlegraphic{\vspace{-16mm}\includegraphics[width=\textwidth,height=3cm]{title}}


\begin{frame}
    \titlepage
    %\vspace{-5mm}
    \begin{flushright}
        \href{http://www.gtcmt.gatech.edu}{\includegraphics[height=.8cm,keepaspectratio]{Logo_GTCMT_black}}
    \end{flushright}
\end{frame}


    \section[intro]{introduction}
        \begin{frame}{lecture content}{overview}
            \begin{enumerate}
                \item   LTI systems and their properties
                \smallskip
                \item   convolution
            \end{enumerate}
        \end{frame}

        \begin{frame}{systems}{introduction}
            a system:
            \begin{itemize}
                \item   any process producing an output signal in response to an input signal
            \end{itemize}
		    \begin{figure}
				\centering
		        \begin{picture}(80,35)
		            %boxes
		            %\put(0,30){\ovalbox{\footnotesize{\parbox{20mm}{\vspace{2mm}\centering{Composer}\vspace{2mm}}}}}
		            \put(30,30){\ovalbox{\footnotesize{\parbox{20mm}{\vspace{2.3mm}\centering{System}\vspace{2.3mm}}}}}
		            %\put(60,30){\ovalbox{\footnotesize{\parbox{20mm}{\vspace{2mm}\centering{Recipient}\vspace{2mm}}}}}
		
		            % horizontal
		            \put(22.4,30.6){\vector(1,0){7.8}}
		            \put(52.4,30.6){\vector(1,0){7.8}}
		            \put(15,30){\text{$x(t)$}}
		            \put(60,30){\text{$y(t)$}}
		        \end{picture}
		    \end{figure}
            \vspace{-25mm}
            \question{name examples for systems in signal processing/the real world}
            
            \begin{itemize}
                \item   filters, effects
                \item   vocal tract
                \item   room
                \item   (audio) cable
                \item   \ldots
            \end{itemize}
        \end{frame}
    \section[LTI]{Linear Time-Invariant systems}
        \begin{frame}{systems}{linearity and non-linearity}
            \begin{columns}
            \column{.7\linewidth}
            \begin{itemize}
                \item   examples for mostly linear systems:
                    \begin{itemize}
                        \item   room
                        \item   eq
                    \end{itemize}
                \bigskip
                \item   examples for non-linear systems:
                    \begin{itemize}
                        \item   diode
                        \item   vacuum tube
                    \end{itemize}
            \end{itemize}
            \column{.3\linewidth}
            \begin{figure}%
                \includegraphics[scale=.75]{graph/linearnonlinear}%
            \end{figure}
            \end{columns}

            \addreference{\url{http://www.sfu.ca/sonic-studio/handbook/Linear.html}}
        \end{frame}

        \begin{frame}{systems}{linearity}
            \toremember{linearity is defined by two properties}
            
            \begin{enumerate}
                \item   \textbf{homogeneity}:
                    \[f(ax) = a f(x)\]
                \bigskip
                \item   \textbf{superposition} (additivity):
                \[f(x+y) = f(x) + f(y)\]
            \end{enumerate}
        \end{frame}
        \begin{frame}{systems}{time invariance}
            \toremember{time invariance}
            
            \begin{itemize}
                \item   does not change with time:
                    \[f\left(x(t-\tau)\right) = f(x)(t-\tau)\]
            \end{itemize}
            \pause
            \bigskip
            \begin{block}{LTI: Linear Time-invariant Systems}
                are a great simplification for many real-world systems we would like to model --- circuits, spring-mass-damper systems, etc.
            \end{block}
        \end{frame}
        \begin{frame}{systems}{LTI system example}
            velocity of mass an a table
            \begin{enumerate}
                \item   hammer gives \textit{impulse}
                \item   system \textit{responds} with velocity
                \bigskip
                \item<2->[]   \textbf{linearity}:\\ double force, double velocity, multiple strikes add up
                \smallskip
                \item<2->[]   \textbf{time invariance}:\\ system reacts the same whether I do it now or tomorrow
            \end{enumerate}
        \end{frame}
        \begin{frame}{systems}{other system characteristics}
            \begin{itemize}
                \item   \textbf{causality}:\\ output depends only on past and present input
                \bigskip
                \item   \textbf{BIBO stability}:\\ output is bounded for bounded input
            \end{itemize}
        \end{frame}
    \section{convolution}
        \begin{frame}{convolution}{introduction}
            \question{we know how a system reacts to an impulse, but what of a more complex input signal}

            \begin{itemize}
                \item<2-> assume that the signal is constructed from many densely packed impulses
                \item<2->[$\Rightarrow$] output is superposition of all individual responses
            \end{itemize}
            \pause
            \bigskip
            \textbf{convolution}
            \begin{equation*}
                y(t) = (x \ast h)(t) := \int\limits_{-\infty}^{\infty}x(\tau)h(t-\tau)d\tau
            \end{equation*}
        \end{frame}
\begin{frame}{convolution}{animation}
            \vspace{-5mm}
            \includeanimation{Convolution}{000}{600}{10}
            %\begin{center}
                 %\animategraphics[autoplay,loop]{10}{animateConvolution/Convolution-}{000}{600}        
            %\end{center}
            %\addreference{matlab source: \href{https://github.com/alexanderlerch/ACA-Slides/blob/master/matlab/animateConvolution.m}{matlab/animateConvolution.m}}
            %\inserticon{video}
\end{frame}
    
\begin{frame}\frametitle{convolution}\framesubtitle{exercise --- convolution by hand}
    compute the convolution of the following two signals
    \begin{figure}%
    \includegraphics[width=.75\columnwidth]{graph/convolution_exercise}%
    \end{figure}
    \bigskip
    steps:
    \begin{enumerate}
        \item   flip one signal
        \item   multiply the two signals
        \item   integrate the result
        \item   shift
        \item   go to 2.
    \end{enumerate}
\end{frame}
    
\section[conv. prop.]{convolution properties}
\begin{frame}\frametitle{convolution}\framesubtitle{identity and impulse response}
    \begin{eqnarray*}
        x(t) &=& \delta(t)\ast x(t) \\
        h(t) &=& \delta(t)\ast h(t) 
    \end{eqnarray*}

    \bigskip
    \begin{itemize}
        \item<2->   describes the response of a system to an impulse as a function of time
        \smallskip
        \item<3->   as an impulse includes all frequencies, the resulting IR defines the response for all frequencies
        \smallskip
        \item<4->   the convolution of $\delta(t)$ with a signal/impulse response results in that impulse response
    \end{itemize}
\end{frame}
    
	\begin{frame}\frametitle{convolution}\framesubtitle{properties}
		\begin{equation*}
			y(t) = x(t) \ast h(t) = \int\limits_{-\infty}^{\infty}{h(\tau)\cdot x(t-\tau)} d\tau
		\end{equation*}
		\begin{itemize}
			\item	\textbf{commutativity}
				\begin{equation*}
					h(t) \ast x(t)	= x(t) \ast h(t)
				\end{equation*}
			\pause
			\item	\textbf{associativity}
				\begin{equation*}
					\big(g(t) \ast h(t)\big) \ast x(t) = g(t) \ast \big(h(t) \ast x(t)\big)
				\end{equation*}
			\pause
			\item	\textbf{distributivity}
				\begin{equation*}
					g(t) \ast \big(h(t) + x(t)\big) = \big(g(t) \ast h(t)\big) + \big(g(t) \ast x(t)\big)
				\end{equation*}
		\end{itemize}
\end{frame}

\section[deriv]{convolution property derivation}

\begin{frame}\frametitle{convolution}\framesubtitle{derivation: commutativity}
\begin{footnotesize}
		\begin{equation*}
			h(t) \ast x(t)	= x(t) \ast h(t) 
		\end{equation*}
		
        \bigskip
        substituting $\tau'=t-\tau$:
		\begin{eqnarray*}
			x(t) \ast h(t)
					&=& \int\limits_{-\infty}^{\infty}{h(\tau)\cdot x(t-\tau)} d\tau\nonumber\\
                    \pause&=& \int\limits_{-\infty}^{\infty}{h(t-\tau')\cdot x(\tau')}d(t-\tau') \nonumber\\
					\pause&=& \int\limits_{-\infty}^{\infty}{x(\tau')\cdot h(t-\tau')}d\tau' 
		\end{eqnarray*}
\end{footnotesize}
\end{frame}	

\begin{frame}\frametitle{convolution}\framesubtitle{derivation: associativity}
\vspace{-6mm}
\begin{footnotesize}
		\begin{equation*}
			\big(g(t) \ast h(t)\big) \ast x(t) = g(t) \ast \big(h(t) \ast x(t)\big)
		\end{equation*}
		changing the order of sums and shifting the operands as shown below:
		\begin{eqnarray*}
			\big(g(t) \ast h(t)\big) \ast x(t)	&=& \int\limits_{\tau=-\infty}^{\infty}{\big(g(\tau) \ast h(\tau)\big)\cdot x(t-\tau)}d\tau\nonumber\\
%												&=& \int\limits_{\tau=-\infty}^{\infty}{\left(\sum\limits_{\xi=-\infty}^{\infty}{g(\xi)\cdot h(\tau-\xi)}\cdot x(t-\tau)\right)}\nonumber\\
												\pause&=& \int\limits_{-\infty}^{\infty}{\int\limits_{-\infty}^{\infty}{g(\xi)\cdot h(\tau-\xi)}\cdot x(t-\tau)}d\tau d\xi\nonumber\\
												\pause&=& \int\limits_{-\infty}^{\infty}{g(\xi)\cdot \int\limits_{-\infty}^{\infty}{h(\tau-\xi)}\cdot x(t-\tau)}d\tau d\xi\nonumber\\
												\pause&=& \int\limits_{-\infty}^{\infty}{g(\xi)\cdot \int\limits_{-\infty}^{\infty}{h(\tau')}\cdot x(t-\xi-\tau')d\tau' d\xi}\nonumber\\
												\pause&=& \int\limits_{-\infty}^{\infty}{g(\xi)\cdot  \big(h(t-\xi) \ast x(t-\xi)\big)} d\xi\nonumber\\
												\pause&=& g(t) \ast \big(h(t) \ast x(t)\big) 
		\end{eqnarray*}
\end{footnotesize}
\end{frame}	

\begin{frame}\frametitle{convolution}\framesubtitle{derivation: distributivity}
\begin{footnotesize}
		\begin{equation*}
			g(t) \ast \big(h(t) + x(t)\big) = g(t) \ast h(t) + g(t) \ast x(t) 
		\end{equation*}
		\begin{eqnarray*}
			g(t) \ast \big(h(t) + x(t)\big) &=& \int\limits_{-\infty}^{\infty}{g(\tau) \cdot\big(h(t-\tau) + x(t-\tau)\big)}d\tau\nonumber\\
									\pause&=& \int\limits_{-\infty}^{\infty}{g(\tau) \cdot h(t-\tau) + g(\tau) \cdot x(t-\tau)}d\tau\nonumber\\
									\pause&=& \int\limits_{-\infty}^{\infty}{g(\tau) \cdot h(t-\tau)}d\tau + \int\limits_{-\infty}^{\infty}{g(\tau) \cdot x(t-\tau)}d\tau\nonumber\\
									\pause&=& g(t) \ast h(t) + g(t) \ast x(t) 
		\end{eqnarray*}
\end{footnotesize}
\end{frame}	

    \section[summary]{summary}
        \begin{frame}{systems}{summary}
            \vspace{-3mm}
            \begin{itemize}
                \item   \textbf{LTI system}: approximation of many real-world systems
                    \begin{itemize}
                    \item<2-> properties:
                        \begin{itemize}
                            \item   linearity 1: homogeneity (scaling)
                            \item   linearity 2: superposition (additivity)
                            \item   time invariance (system doesn't change)
                            \item<3->   causality (no future input)
                            \item<3->   BIBO --- bounded input bounded output
                        \end{itemize}
                    \end{itemize}
                \bigskip
                \item<4-> \textbf{impulse response} is a complete description of an LTI system
                \bigskip
                \item<5-> \textbf{convolution}: describes process of generating output of LTI system from input
                    \begin{itemize}
                    \item<6-> properties:
                        \begin{itemize}
                            \item   commutative
                            \item   associative
                            \item   distributive
                        \end{itemize}
                    \end{itemize}
            \end{itemize}
        \end{frame}

    
\end{document}

