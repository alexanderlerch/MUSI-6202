% move all configuration stuff into one file so we can focus on the content
\input{../shared/common}
\input{../shared/definitions}


\subtitle{Part 9: Fast Convolution}

%%%%%%%%%%%%%%%%%%%%%%%%%%%%%%%%%%%%%%%%%%%%%%%%%%%%%%%%%%%%%%%%%%%%%%%%%%%%
\begin{document}
    % generate title page
	\title[]{Digital Signal Processing for Music}   
\author[alexander lerch]{alexander lerch} 
%\institute{~}
%\date[Alexander Lerch]{}
\titlegraphic{\vspace{-16mm}\includegraphics[width=\textwidth,height=3cm]{title}}


\begin{frame}
    \titlepage
    %\vspace{-5mm}
    \begin{flushright}
        \href{http://www.gtcmt.gatech.edu}{\includegraphics[height=.8cm,keepaspectratio]{Logo_GTCMT_black}}
    \end{flushright}
\end{frame}


\section[intro]{introduction}
	\begin{frame}{fast convolution}{introduction}
		convolution: measure impulse response h(i) and apply FIR filter to signal
		\begin{eqnarray*}
			y(i) &=& x(i) \ast h(i)\\
				 &=& \sum\limits_{j=-\infty}^{\infty}{h(j)\cdot x(i-j)}\\
			Y(z) &=& X(z) \cdot H(z)
		\end{eqnarray*}
	\end{frame}

\section[DFT]{DFT convolution}
	\begin{frame}{DFT convolution}{signal and impulse response 1/2}
		\begin{itemize}
			\item	multiplication: length of $H(z) = M$ must equal length of $X(z) = N$
			\item	minimum DFT length: $L \leq M+N-1$
		\end{itemize}
		\only<1>{
		\begin{figure}
			\centering
				\includegraphics[scale=.4]{graph/conv_dft_zp}
		\end{figure}
		}
		\only<2->{
		\begin{enumerate}
			\item	$X = DFT(x'(i))$
			\item	$H = DFT(h'(i))$
			\item	$Y = X\cdot H$
			\item	$y = DFT^{-1}(Y)$
			\item   throw away zeros if DFT was longer than $M+N$
		\end{enumerate}
		}
		\only<3>{
		\begin{figure}
			\centering
				\includegraphics[scale=.4]{graph/conv_dft_res}
		\end{figure}
		}
		\vspace{50mm}
	\end{frame}

	\begin{frame}{DFT convolution}{signal and impulse response 2/2}
		\textbf{properties}:
		\begin{itemize}
			\item	no real-time: signal has to be known completely
			\item	high memory requirements (signal length $N$ + impulse response length $M$)
			\begin{itemize}
				\item	when FFT: next larger power of two
			\end{itemize}
		\end{itemize}
	\end{frame}

\section[blocked]{blocked convolution}
	\begin{frame}{blocked convolution}{blocked signal and impulse response 1/2}
		\begin{enumerate}
			\item	split input signal into blocks of length $M$
			\item	DFT convolution with each block (zeropadding)
			\item	overlap and save
		\end{enumerate}
		\only<1>{
		\begin{figure}
			\centering
				\includegraphics[scale=.4]{graph/conv_dft_split}
		\end{figure}
		}
		\only<2>{
		\begin{figure}
			\centering
				\includegraphics[scale=.4]{graph/conv_block}
		\end{figure}
		}
		\only<3>{
		\begin{figure}
			\centering
				\includegraphics[scale=.4]{graph/conv_overlapsave}
		\end{figure}
		}
		\only<4>{
		\begin{figure}
			\centering
				\includegraphics[scale=.5]{graph/conv_block2}
		\end{figure}
		}
		\vspace{50mm}
	\end{frame}

	\begin{frame}{blocked convolution}{blocked signal and impulse response 2/2}
		\textbf{properties}:
		\begin{itemize}
			\item	minimum latency: impulse response length
			\item	long FFT, but more efficient
			\item	FFT of impulse response \textit{is only computed once}
		\end{itemize}
	\end{frame}

\section[partitioned]{partitioned convolution}
	\begin{frame}{partitioned convolution}{blocked signal and blocked impulse response 1/3}
		\begin{enumerate}
			\item	split \textbf{both} input signal and impulse response into blocks of arbitrary length
			\item	DFT convolution with each signal block with each impulse response block (zeropadding)
			\item	overlap and save
		\end{enumerate}
		\only<1>{
		\begin{figure}
			\centering
				\includegraphics[scale=.4]{graph/conv_blockblock}
		\end{figure}
		}
		\only<2>{
		\begin{figure}
			\centering
				\includegraphics[scale=.4]{graph/conv_fast}
		\end{figure}
		}
		\only<3>{
		\begin{figure}
			\centering
				\includegraphics[scale=.4]{graph/conv_overlapsave}
		\end{figure}
		}
		\vspace{50mm}
	\end{frame}

	\begin{frame}{partitioned convolution}{blocked signal and blocked impulse response 2/3}
		\vspace{-5mm}
		\begin{figure}
			\centering
				\includegraphics[scale=.35]{graph/conv_fast2}
		\end{figure}
	\end{frame}

	\begin{frame}{partitioned convolution}{blocked signal and blocked impulse response 3/3}
		\textbf{properties}:
		\begin{itemize}
			\item	arbitrary choice of latency/FFT length 
				\begin{itemize}
					\item	long FFT: high latency, low workload
					\item	short FFT: short latency, high workload
				\end{itemize}
			\item	FFTs of IR computed only once
		\end{itemize}
	\end{frame}

\section[non-uniform]{non-uniform partitioned convolution}
	\begin{frame}{non-uniform partitioned convolution}{different block lengths}
		\begin{itemize}
			\item	fast convolution: latency still formidable for efficient implementation
			\pause
			\item[$\Rightarrow$] \textbf{non-uniform block lengths}
		\end{itemize}
			\begin{figure}
			\centering
				\includegraphics[scale=.4]{graph/conv_nonuniform}
		\end{figure}
		\pause
		\begin{itemize}
			\item   \textbf{advantages}:
				\begin{itemize}
					\item   \textit{any} desirable latency
				\end{itemize}
			\item   \textbf{disadvantages}:
				\begin{itemize}
					\item   less efficient due to multiple FFT lengths (but: inefficiency of short FFT partly compensated by very long FFTs)
					\item   complex implementation
					\item   comparably high memory usage (IR in many different FFT lengths)
				\end{itemize}
		\end{itemize}

	\end{frame}

    %\section[summary]{summary}
            %\begin{frame}{Fourier transform}{summary 1/2}
                %\textbf{FT properties}
                %\begin{enumerate}
                    %\item   invertibility
                    %\item   linearity
                    %\item   convolution --- multiplication
                    %\item   Parseval's theorem
                    %\item   time shift --- phase shift
                    %\item   symmetry
                    %\item   time scaling --- frequency scaling
                %\end{enumerate}
            %\end{frame}	
    %
            %\begin{frame}{Fourier transform}{summary 2/2}
                %\begin{enumerate}
                    %\item   Fourier series can describe any periodic function $\rightarrow$ discrete ``spectrum''
                    %\item   continuous FT transforms any continuous function $\rightarrow$ continuous spectrum
                    %\item   STFT transforms a segment of the signal $\rightarrow$ convolution with window spectrum
                    %\item   FT of sampled signals $\rightarrow$ periodic
                    %\item   DFT $\rightarrow$ sampled FT of periodic continuation
                %\end{enumerate}
                    %\pause
                %\begin{itemize}
                    %\item   \textbf{spectrum is periodic $\leftrightarrow$ time signal is discrete}
                    %\item   \textbf{spectrum is discrete $\leftrightarrow$ time signal is periodic}
                %\end{itemize}
            %\end{frame}	
    
\end{document}

