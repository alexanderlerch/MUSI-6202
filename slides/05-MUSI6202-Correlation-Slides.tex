% move all configuration stuff into one file so we can focus on the content
\input{../shared/common}
\input{../shared/beamersetup}
%%%%%%%%%%%%%%%%%%%%%%%%%%%%%%%%%%%%%%%%%%%%%%%%%%%%%%%%%%%%%%%%%%%%%%%%%%%%%%%%%%
%%%%%%%%%%%%%%%%%%%%%%%%%%%%%%%%%%%%%%%%%%%%%%%%%%%%%%%%%%%%%%%%%%%%%%%%%%%%%%%%%%
% title information
\title[]{MUSI6202: Digital Signal Processing for Music}   
\author[alexander lerch]{alexander lerch} 
%\institute{~}
%\date[Alexander Lerch]{}
%\titlegraphic{\vspace{-16mm}\includegraphics[width=\textwidth,height=3cm]{title}}

\input{../shared/definitions}



\subtitle{Part 5: Signal Similarity --- Correlation}

%%%%%%%%%%%%%%%%%%%%%%%%%%%%%%%%%%%%%%%%%%%%%%%%%%%%%%%%%%%%%%%%%%%%%%%%%%%%
\begin{document}
    % generate title page
	\title[]{Digital Signal Processing for Music}   
\author[alexander lerch]{alexander lerch} 
%\institute{~}
%\date[Alexander Lerch]{}
\titlegraphic{\vspace{-16mm}\includegraphics[width=\textwidth,height=3cm]{title}}


\begin{frame}
    \titlepage
    %\vspace{-5mm}
    \begin{flushright}
        \href{http://www.gtcmt.gatech.edu}{\includegraphics[height=.8cm,keepaspectratio]{Logo_GTCMT_black}}
    \end{flushright}
\end{frame}


\section{introduction}
\begin{frame}{signal similarity}{introduction}
    \begin{itemize}
        \item   correlation function
            \begin{itemize}
                \item   indicates (linear) dependencies between two signals
                \item   shifts the signals to find the dependency for each shift in time
            \end{itemize}
        \bigskip
        \item lecture content
            \begin{itemize}
                \item (cross) correlation function (CCF)
                \item   auto correlation function (ACF)
            \end{itemize}
            
    \end{itemize}
\end{frame}

\section{CCF}

\begin{frame}{signal similarity}{correlation function}
	compute similarity between two \textbf{stationary} signals $x$,$y$
    \begin{equation*}
        r_\mathrm{xy}(\tau)=\mathcal{E}\lbrace x(t)y(t+\tau)\rbrace  
    \end{equation*}  
	\pause
    
	\begin{itemize}
		\item	\textbf{continuous}:
            \begin{equation*}
                r_\mathrm{xy}(\tau) = \int\limits_{-\infty}^{\infty}{x(t)\cdot y(t+\tau)dt}
            \end{equation*}
		\item	\textbf{discrete}:
            \begin{equation*}
                r_\mathrm{xy}(\eta) = \sum\limits_{i=-\infty}^{\infty}{x(i)\cdot y(i+\eta)}
            \end{equation*}
	\end{itemize}
\end{frame}

\begin{frame}{signal similarity}{correlation function: animation}
    \vspace{-5mm}
    \begin{footnotesize}
    \begin{equation*}
        r_\mathrm{xy}(\tau) = \int\limits_{-\infty}^{\infty}{x(t)\cdot y(t+\tau)}dt
    \end{equation*}
    \end{footnotesize}
    \includeanimation
        {Correlation}
        {000}
        {250}
        {10}
\end{frame}

\begin{frame}{signal similarity}{correlation function: use cases}
    \begin{itemize}
        \item   find (linear!) similarity between two signals (e.g., clean and noisy)
        \item   find time shift between to similar signals
        \bigskip
        \item<2->   example: \textbf{radar}
            \begin{itemize}
                \item   correlate sent signal with received signal
                \item   pick maximum location and convert to distance of object
            \end{itemize}
    \end{itemize}
    \only<2>{
    \includeanimationnomatlab
        {radar}
        {00}
        {43}
        {10}
        {\href{https://en.wikipedia.org/wiki/Radar}{en.wikipedia.org/wiki/Radar}}
    }    
\end{frame}

\section[correlation coefficient]{normalized correlation and correlation coefficient}
\begin{frame}{signal similarity}{normalization \& correlation coefficient}
    \begin{itemize}
        \item normalization
        \begin{equation*}\nonumber
            r_\mathrm{xy}(\tau) = \frac{\mathcal{E}\lbrace(X-\mu_X)(Y-\mu_Y)\rbrace}{\sigma_X\sigma_Y}
        \end{equation*}
        \item   special case: \textbf{Pearson Correlation Coefficient} $r_\mathrm{xy}(0)$ after normalization
    \end{itemize}
    
    
    \bigskip
    \bigskip
    \question{What are possible reasons for normalization}
        \begin{itemize}
            \item   ensuring that function will always be between -1 and 1
            \item   shifting and scaling one signal will not change the coefficient
        \end{itemize}
\end{frame}

\section{possible issues}
\begin{frame}{signal similarity}{correlation function: problems as summary statistic}
    Anscombe's quartet:
    \begin{columns}
    \column{.5\linewidth}
    \begin{itemize}
        \item descriptive statistics
            \begin{itemize}
                \item   identical mean: 7.5
                \item   identical variance: 4.2
                \item   identical \textbf{Pearson correlation coefficient}: 0.816
            \end{itemize}
    \end{itemize}
    \column{.5\linewidth}
    \begin{figure}
        \includegraphics[scale=.2]{graph/Anscombes_quartet}
    \end{figure}
    \end{columns}
    \addreference{\href{https://en.wikipedia.org/wiki/Anscombe's_quartet}{en.wikipedia.org/wiki/Anscombe's\_quartet}}
\end{frame}

\section{examples}
\begin{frame}\frametitle{correlation function}\framesubtitle{examples 1/2}
		\question{Describe the (cross) correlation function between the following signals}
			
		\begin{itemize}
			\item	rectangular window vs.
			\item	windowed sine vs.
			\item	noise
		\end{itemize}
\end{frame}

\begin{frame}\frametitle{correlation function}\framesubtitle{examples 2/2}
    \figwithmatlab{Correlation}
\end{frame}

%\begin{frame}\frametitle{correlation function}\framesubtitle{animation}
    %\vspace{-5mm}
    %\begin{footnotesize}
            %\begin{equation*}
                %%r_\mathrm{xy}(\tau) &=& \int\limits_{-\infty}^{\infty}{x(t)\cdot y(t+\tau)dt}\\
                %r_\mathrm{xy}(\eta) = \sum\limits_{i=-\infty}^{\infty}{x(i)\cdot y(i+\eta)}
            %\end{equation*}
    %\end{footnotesize}
    %%\begin{center}
        %%\animategraphics[loop]{10}{animateCorrelation/Correlation-}{000}{250}        
    %%\end{center}
    %%\addreference{matlab source: \href{https://github.com/alexanderlerch/ACA-Slides/blob/master/matlab/animateCorrelation.m}{matlab/animateCorrelation.m}}
    %\inserticon{video}
%\end{frame}	
%
\begin{frame}{correlation function}{blocked correlation: animation}
    \includeanimation
        {BlockedCorrelation}
        {001}
        {1023}
        {24}
\end{frame}	 

\section{ACF}
\begin{frame}{correlation function}{auto correlation function}
		\begin{equation*}
			r_\mathrm{xx}(\tau)=\mathcal{E}\lbrace x(t)x(t+\tau)\rbrace  
		\end{equation*}

		\pause
		\begin{block}{autocorrelation function properties}
			\begin{itemize}
				\item	\textbf{power}: $r_{xx}(0) = 	\mathcal{E}\lbrace X^2\rbrace $ 

				\smallskip
				\item<2->	\textbf{symmetry} $r_{xx}(\tau)=r_{xx}(-\tau)$\\
					(substitute $t=t'+\tau$)

				\smallskip
                \item<3->	\textbf{global max}: $r_{xx}(\tau)\leq r_{xx}(0)$ 
					%(Binomial theorem $E\left\lbrace \big(x(t)x(t-\tau)\big)^2\right\rbrace $ )

				\smallskip
				\item<4->	\textbf{periodicity}:\\
					The {ACF} of a periodic signal is periodic (period length of input signal)

			\end{itemize}	
		\end{block}
\end{frame}	

%\begin{frame}{correlation and power spectral density}{PSD}
	%\begin{itemize}
		%\item	\textbf{problem}: how to calculate the spectrum of a random process?
				%\begin{itemize}
					%\item	no analytic function to integrate
				%\end{itemize}
				%
		%\pause
		%\item	\textbf{solution}: Wiener-Khinchin theorem
		%\pause
		%\begin{equation*}
			%S_{xx}(\omega) = \mathfrak{F}\lbrace \varphi_{xx}(\tau)\rbrace 
		%\end{equation*}
	%\end{itemize}
%\end{frame}
%
%\begin{frame}{correlation and power spectral density}{PSD: properties}
%
		%\begin{block}{PSD properties}
			%\begin{itemize}
				%\item	\textbf{real} (symmetry in time domain!) 
%
				%\pause
				%\item	\textbf{power preservation}
					%\begin{equation*}
						%\mathcal{E}\lbrace x^2\rbrace  = \varphi_{xx}(\tau = 0) = \frac{1}{2\pi}\int\limits_{-\infty}^{\infty}{S_{xx}(\omega)\enspace d\omega} 
					%\end{equation*}
			%\end{itemize}	
		%\end{block}
%\end{frame}	
%
%\begin{frame}{correlation and power spectral density}{PSD: discrete signals}
	%\begin{itemize}
		%\item	\textbf{correlation}
				%\begin{equation*}
					%\varphi_{xx}(l) = \mathcal{E}\lbrace x(n)\cdot x(n-l)\rbrace 
				%\end{equation*}
		%\pause
		%\item	\textbf{PSD}
				%\begin{equation*}
					%\varphi_{xx}(l) = \mathfrak{F^{-1}}\lbrace S_{xx}(\Omega)\rbrace =\frac{1}{2\pi}\int\limits_{-\pi}^{+\pi}S_{xx}(\Omega)e^{j\Omega l}\enspace d\Omega
				%\end{equation*}  
		%\pause
		%\item	\textbf{power}
				%\begin{equation*}
					%\varphi_{xx}(0)=\frac{1}{2\pi}\int\limits_{-\pi}^{+\pi}S_{xx}(\Omega)\enspace d\Omega
				%\end{equation*}
	%\end{itemize}
%\end{frame}	


\section{summary}
\begin{frame}{correlation function}{summary}
    \begin{itemize}
        \item   correlation function is useful tool to
            \begin{itemize}
                \item   \textit{determine the similarity} between two signals (CCF)
                \item   \textit{identify a shift/latency} between two similar signals (CCF)
                \item   \textit{identify periodicity} vs. noisiness in a signal (ACF)
            \end{itemize}
        \bigskip
        \item<2->   continues to be standard approach for all applications related to the above tasks
        \bigskip
        \item<3->   note: CCF/ACF only convey similarity/periodicity information, no absolute time or phase information (lost in integration)
    \end{itemize}
\end{frame}		

    
\end{document}

