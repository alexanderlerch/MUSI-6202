% move all configuration stuff into one file so we can focus on the content
\input{../shared/common}
\input{../shared/beamersetup}
%%%%%%%%%%%%%%%%%%%%%%%%%%%%%%%%%%%%%%%%%%%%%%%%%%%%%%%%%%%%%%%%%%%%%%%%%%%%%%%%%%
%%%%%%%%%%%%%%%%%%%%%%%%%%%%%%%%%%%%%%%%%%%%%%%%%%%%%%%%%%%%%%%%%%%%%%%%%%%%%%%%%%
% title information
\title[]{MUSI6202: Digital Signal Processing for Music}   
\author[alexander lerch]{alexander lerch} 
%\institute{~}
%\date[Alexander Lerch]{}
%\titlegraphic{\vspace{-16mm}\includegraphics[width=\textwidth,height=3cm]{title}}

\input{../shared/definitions}



\subtitle{Part 10: Discretization 1---Sampling}

%%%%%%%%%%%%%%%%%%%%%%%%%%%%%%%%%%%%%%%%%%%%%%%%%%%%%%%%%%%%%%%%%%%%%%%%%%%%
\begin{document}
    % generate title page
	\title[]{Digital Signal Processing for Music}   
\author[alexander lerch]{alexander lerch} 
%\institute{~}
%\date[Alexander Lerch]{}
\titlegraphic{\vspace{-16mm}\includegraphics[width=\textwidth,height=3cm]{title}}


\begin{frame}
    \titlepage
    %\vspace{-5mm}
    \begin{flushright}
        \href{http://www.gtcmt.gatech.edu}{\includegraphics[height=.8cm,keepaspectratio]{Logo_GTCMT_black}}
    \end{flushright}
\end{frame}


\section[intro]{introduction}
    \begin{frame}\frametitle{sampling and quantization}\framesubtitle{introduction}
        digital signals can only be represented with a limited number of values
        \pause
        
        $\Rightarrow$
        \begin{itemize}
            \item	time discretization:\\ \textcolor{gtgold}{\textbf{sampling}}
            \bigskip
            \item	amplitude discretization:\\ \textbf{quantization}
        \end{itemize}
    \end{frame}
   
	\begin{frame}\frametitle{sampling and quantization}\framesubtitle{sampling}
			\vspace{-5mm}
			\begin{equation*}\nonumber
				T_{\mathrm{S}} = \frac{1}{f_{\mathrm{S}}} 
			\end{equation*}
			\begin{figure}
				\centering
					\includegraphics[scale=.6]{graph/sampling}
				\label{fig:sampling}
			\end{figure}
			\pause
            \vspace{-3mm}
			typical sample rates
			\begin{itemize}
				\item	\unit[8-16]{kHz}: speech (phone)
				\item	\unit[44.1-48]{kHz}: (consumer) audio/music
				\item	higher: production audio
			\end{itemize}
		\end{frame}	

	\section{sampling ambiguity}	
		\begin{frame}\frametitle{sampling and quantization}\framesubtitle{sampling ambiguity 1/4}
            %\includeanimation{graph/samplingambi/samplingambi}{1}{7}{1}
			\begin{center}
				\animategraphics[scale=.8,step]{1}{graph/samplingambi/samplingambi-}{1}{7}        
			\end{center}
			%\begin{figure}
				%\centering
					%\includegraphics[scale=1.0]{Graph/sampling2}
			%\end{figure}
		\end{frame}
		
		\begin{frame}\frametitle{sampling and quantization}\framesubtitle{sampling ambiguity 2/4}
			\begin{eqnarray*}
				f_0 &=& [\unit[1, 5, 7]{kHz}]\\
				f_{\mathrm{S}} &=& \unit[6]{kHz}
			\end{eqnarray*}
            \figwithmatlab{Sampling02}
			%\begin{figure}
				%\centering
					%\includegraphics[scale=.7]{graph/samplingambig}
					%\label{fig:samplingambig}
			%\end{figure}
		\end{frame}	
		
		\begin{frame}\frametitle{sampling and quantization}\framesubtitle{sampling ambiguity 3/4}
			\textbf{wagon wheel effect}
			\invisible<2->{
			\begin{figure}
				\begin{center}
					\includegraphics[scale=.4]{Graph/Stagecoach-Western}
				\end{center}
			\end{figure}
			}
			\visible<2->{
				\vspace{-60mm}
				\begin{figure}
					\begin{center}
						\includegraphics[scale=.2]{Graph/Stagecoach-Western}
					\end{center}
				\end{figure}
				\begin{enumerate}
					\item	$f_{wheel} < \frac{f_S}{2}$: speeding up
					\pause
					\item	$\frac{f_S}{2} < f_{wheel} < f_S$: slowing down
					\pause
					\item	$f_{wheel} = f_S$: standing still
					\pause
					\item	$\ldots$
				\end{enumerate}
			}
		\end{frame}

		\begin{frame}\frametitle{sampling and quantization}\framesubtitle{sampling ambiguity 4/4}
			\url{http://youtu.be/uENITui5_jU}
		\end{frame}	
        
        
		\begin{frame}\frametitle{sampling and quantization}\framesubtitle{sampling}
			\setbeamercovered{invisible}
            \only<1-2>{
            \begin{columns}
            \column{.4\linewidth}
                \begin{itemize}
                    \item[]	$x(t) \mapsto X(\jom)$
                    \pause
                    \item[]	$x(t)\cdot \delta_T \mapsto X(\jom)\ast \delta_{\omega_T}$
                \end{itemize}		
            \column{.6\linewidth}
                \figwithmatlab{SpecSampling}
            \end{columns}
            }
            \only<3->{
                    \includeanimation{Aliasing}{01}{48}{10}

			%\begin{center}
				%\animategraphics[scale=.5,autoplay,loop]{10}{graph/SpectralAliasing/aliasing_}{1}{101}        
			%\end{center}
            }
		\end{frame}
		
	\section{sampling theorem}	
		\begin{frame}\frametitle{sampling and quantization}\framesubtitle{sampling theorem}
			\toremember{}
			\begin{block}{sampling theorem}
				\centering
				A sampled audio signal can  be reconstructed \textbf{without loss of information} if the sample rate $f_{\mathrm{S}}$ is higher than twice the bandwidth $f_{\mathrm{max}}$  of the signal.
				\begin{equation*}\label{eq:sample_theorem}	
					f_{\mathrm{S}} > 2\cdot f_{\mathrm{max}}
				\end{equation*}
			\end{block}
			%\begin{quote}
				 %``The intuitive justification is that, if x(t) contains no frequencies higher than \nicefrac{f_\mathrm{max}, it cannot change to a substantially new value in a time less than one-half cycle of the highest frequency".
			%\end{quote}
		\end{frame}
		
	\section{aliasing}	
		\begin{frame}\frametitle{sampling and quantization}\framesubtitle{sampling: aliasing}
			\begin{figure}
				\begin{center}
					\includegraphics[scale=0.5]{Graph/aliasing}
				\end{center}
			\end{figure} 
		\end{frame}	

		\begin{frame}\frametitle{sampling and quantization}\framesubtitle{sampling: aliasing examples 1/2}
			\vspace{-3mm}
			audio example: sinesweep 100--10k at 24, 12, 6k
            \vspace{-3mm}
			\begin{columns}
				\column{.7\textwidth}
				\vspace{-5mm}
				\begin{figure}
					\includegraphics[scale=.35]{graph/sinealiasing_1}
				\end{figure}
				\visible<2->{
				\vspace{-8mm}
				\begin{figure}
					\includegraphics[scale=.35]{graph/sinealiasing_2}
				\end{figure}
				}
				\visible<3->{
				\vspace{-8mm}
				\begin{figure}
					\includegraphics[scale=.35]{graph/sinealiasing_3}
				\end{figure}
				}
				\column{.3\textwidth}
				
				\includeaudio{sinealiasing_1}
				%\begin{figure}
					%\includegraphics[scale=.05]{graph/SpeakerIcon}
				%\end{figure}
				%\vspace{-3mm}
				%sinealiasing\_1.wav
				\bigskip
				\bigskip
				\bigskip
				\bigskip
				\bigskip
				
				\visible<2->{
				\includeaudio{sinealiasing_2}
				%\begin{figure}
					%\includegraphics[scale=.05]{graph/SpeakerIcon}
				%\end{figure}
				%\vspace{-3mm}
				%sinealiasing\_2.wav
				\bigskip
				\bigskip
				\bigskip
				\bigskip
				\bigskip
			   }
				
				\visible<3->{
				\includeaudio{sinealiasing_3}
				%\begin{figure}
					%\includegraphics[scale=.05]{graph/SpeakerIcon}
				%\end{figure}
				%\vspace{-3mm}
				%sinealiasing\_3.wav
				}
			\end{columns}
			%\begin{itemize}
				%\item	original (\unit[44.1]{kHz})
						%\begin{figure}
							%\includemovie[poster=graph/SpeakerIcon.png,mouse=true]{1cm}{1cm}{audio/bigband.wav}
						%\end{figure}
				%\item	downsampled (\unit[11.025]{kHz}) \textit{without} aliasing filter
						%\begin{figure}
							%\includemovie[poster=graph/SpeakerIcon.png,mouse=true]{1cm}{1cm}{audio/bigbandds.wav}
						%\end{figure}
				%\item	downsampled (\unit[11.025]{kHz}) \textit{with} aliasing filter
						%\begin{figure}
							%\includemovie[poster=graph/SpeakerIcon.png,mouse=true]{1cm}{1cm}{audio/bigbandds_proper.wav}
						%\end{figure}
			%\end{itemize}
		\end{frame}	

		\begin{frame}\frametitle{sampling and quantization}\framesubtitle{sampling: aliasing examples 2/2}
            \setbeamercovered{invisible}
            \begin{itemize}
                \item \textbf{bigband}
                \begin{itemize}
                    \item   original (\unit[48]{kHz}): \includeaudio{bigband}
                    \item   samples discarded (\unit[6]{kHz}): \includeaudio{bigbandds8}
                    \item   downsampling  w/ anti-aliasing filter (\unit[6]{kHz}): \includeaudio{bigbandds8_proper}
                \end{itemize}
                \pause
                \bigskip
                \item   \textbf{sax}
                \begin{itemize}
                    \item   original (\unit[48]{kHz}): \includeaudio{alto-sax}
                    \item   samples discarded (\unit[6]{kHz}): \includeaudio{alto-saxds8}
                    \item   downsampling w/ anti-aliasing filter (\unit[6]{kHz}): \includeaudio{alto-saxds8_proper}
                \end{itemize}
            \end{itemize}
		\end{frame}

	\section{summary}	
		\begin{frame}{sampling and quantization}{sampling: summary 1/2}
			\begin{enumerate}
				\item[]   continuous input signal
                \item[] \hspace{10mm}$\downarrow$
				\pause
				\item   \textbf{anti-aliasing filter}
				\pause
                \smallskip
				\item[] filtered continuous input signal
				\pause
				\item   \textbf{sampling}
				\pause
                \smallskip
				\item[] sampled input signal
				\pause
                \smallskip
				\item   \textbf{reconstruction filter}
				\pause
                \item[] \hspace{10mm}$\downarrow$
				\item[] continuous output signal
			\end{enumerate}
		\end{frame}
 
		\begin{frame}{sampling and quantization}{sampling: summary 2/2}
			\begin{itemize}
				\item   \textbf{sampling theorem}
                    \begin{quote}
                        A sampled audio signal can be reconstructed without loss of information if the sample rate $f_\mathrm{S}$ is higher than twice the bandwidth f max of the signal.
                    \end{quote}
                    \smallskip
                    \begin{itemize}
                        \item<2->   perfect reconstruction!
                        \item<2->   ensure accordance through filtering, otherwise aliasing (mirror frequencies)
                    \end{itemize}
                \bigskip
                \item<3->   band of interest does not have to be base band ($0\ldots f_\mathrm{S}/2$), but any band ($k\cdot f_\mathrm{S}/2\ldots(k+1)\cdot f_\mathrm{S}/2$) as long as the bandwidth is not wider 
			\end{itemize}
		\end{frame}

\end{document}

