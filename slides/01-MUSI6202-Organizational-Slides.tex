% move all configuration stuff into one file so we can focus on the content
\input{../shared/common}
\input{../shared/definitions}

\subtitle{Part 1: Organizational}

%%%%%%%%%%%%%%%%%%%%%%%%%%%%%%%%%%%%%%%%%%%%%%%%%%%%%%%%%%%%%%%%%%%%%%%%%%%%
\begin{document}
    % generate title page
	\title[]{Digital Signal Processing for Music}   
\author[alexander lerch]{alexander lerch} 
%\institute{~}
%\date[Alexander Lerch]{}
\titlegraphic{\vspace{-16mm}\includegraphics[width=\textwidth,height=3cm]{title}}


\begin{frame}
    \titlepage
    %\vspace{-5mm}
    \begin{flushright}
        \href{http://www.gtcmt.gatech.edu}{\includegraphics[height=.8cm,keepaspectratio]{Logo_GTCMT_black}}
    \end{flushright}
\end{frame}


    \section[contact]{contact \& resources}
        \begin{frame}\frametitle{organizational}\framesubtitle{links \& contact}
            \begin{itemize}
                \item \textbf{contact info}
                    \begin{itemize}
                        \item   \textit{alexander lerch}
                            \begin{itemize}
                                \item   {email}: \url{mailto:alexander.lerch@gatech.edu}
                                \item   {www}: \url{www.audiocontentanalysis.org}
                                \item   {office}: Couch 205B
                                \item   {office hours}:  \textbf{by appointment}: \url{https://www.calendly.com/alexanderlerch}
                            \end{itemize}
                        %\item   \textit{teaching assistants}
                            %\begin{itemize}
                                %\item    Siddharth Gururani: \url{siddgururani@gatech.edu}
                                %\item    Ashis Pati: \url{ashis.pati@gatech.edu}
                            %\end{itemize}
                    \end{itemize}
                
                \smallskip
                \item<2-> \textbf{classes}
                    \begin{itemize}
                        \item   Mon, Wed  3:00--4:15pm in WV163
                        \item   additional \textit{tutorial group}: TBD
                    \end{itemize}
                
                \smallskip
                \item<3-> \textbf{class resources}
                    \begin{itemize}
                        \item	\textit{canvas}:
                            \begin{itemize}
                                \item   syllabus, grades, slides: \url{www.canvas.gatech.edu}
                            \end{itemize}
                    \end{itemize}
            \end{itemize}
        \end{frame}

    \section{course intro}
        \begin{frame}\frametitle{organizational}\framesubtitle{goals \& requirements}
            \begin{itemize}
                \item	\textbf{goals}
                        \begin{enumerate}
                            \item   ability to comprehend typical representations of digital systems such as block diagrams and difference equations,
                            \item   understanding of typical transforms in DSP such as the Fourier transform or the Z-transform,
                            \item    ability to use this understanding to design audio processing systems such as audio effects, and
                            \item	ability to implement such designs in a programming language such as Matlab.
                        \end{enumerate}

                \smallskip
                \item<2-> \textbf{requirements}	
                        \begin{itemize}
                            \item	math
                            \item	rudimentary programming skills, familiarity with Matlab
                        \end{itemize}
            \end{itemize}
        \end{frame}
        \begin{frame}\frametitle{organizational}\framesubtitle{outline}
            \vspace{-3mm}
            \begin{scriptsize}
                \begin{tabular}{l|p{.4\linewidth}|p{.1\linewidth}|p{.2\linewidth}||p{.1\linewidth}}
            \textbf{date} & \textbf{topics} & \textbf{exercise} & \textbf{assignment} & \textbf{notes}\\
            \hline\hline
            01/07 & introduction, signals, periodicity, random processes, pdf, moments, correlation & correlation &  &  \\
            01/14 & convolution, power spectral density & FIR filter& filter \& convolution & \\
            01/21 & Fourier series \& Fourier transform& DFT & Fourier analysis & MLK Hldy.\\
            01/28 &Fourier transform, sampling, quantization, SNR, number formats& quantization&& \\
            02/04 &oversampling, dither, noise-shaping, non-linear quant.&&dither, ns  & \\
            02/11 &z-transform, digital audio filters, FIR/IIR, FFT filtering& biquad& & midterm I\\
            02/18 &sample rate conversion, real-time systems &resampling&& \\
            02/25 &delay-based FX and reverb&vibrato& mod. fx&\\
            03/04 &dynamics processing& PPM & limiter & guthman\\
            03/11 &time-segment processing (OLA)&ola&& midterm II \\
            03/18 &&&&spring break \\
            03/25 &phase-vocoder&& phase voc&\\
            04/01 &source coding: LPC, ADPCM&&& \\
            04/08 &source coding: Huffman, AAC&&& \\
            04/15 &denoising&&& \\
            04/22 &competition results&&& \\
                \end{tabular}
            \end{scriptsize}
        \end{frame}

    \section{materials}
        \begin{frame}\frametitle{organizational}\framesubtitle{course materials}
        \begin{itemize}
            \item   \textbf{roughly based on}:  
                \begin{itemize}
                    \item   Z\"olzer, Udo (2008): \textit{Digital Audio Signal Processing}, Wiley 
                \end{itemize}
             \pause
             \bigskip
             \item   \textbf{additional  reading}:  
                \begin{itemize}
                    \item   Lyon, Richard (2011): \textit{Understanding Digital Signal Processing}, Prentice Hall 
                    \item   Z\"olzer, Udo (2011): \textit{DAFX: Digital Audio Effects}, Wiley
                \end{itemize}
            \pause
            \bigskip
            \item   \textbf{{additional} additional reading}:
                \begin{itemize}
                    \item   Pohlmann, Ken (2000): Principles of Digital Audio, 4th, McGraw-Hill
                    \item   Watkinson, John (2001): The Art of Digital Audio, Focal Press
                \end{itemize}

                %\only<1-2>{\vspace{5mm}
                %\begin{columns}[c]
                    %\column{.1\textwidth}
                    %\column{.3\textwidth}\includegraphics[scale=0.005]{graph/psychologyofhearing}
                    %\column{.3\textwidth}\includegraphics[scale=0.005]{graph/psychologyofmusic}
                    %\column{.3\textwidth}\includegraphics[scale=0.25]{graph/musicophilia}
                %\end{columns}}
            \pause
            \bigskip
            \item   \textbf{software}: 
                \begin{itemize}
                    \item   Matlab: \url{www.matlab.gatech.edu}
                    \item   github.com etc
                \end{itemize}
        \end{itemize}
       \end{frame}

    \section{assessment}
        \begin{frame}\frametitle{organizational}\framesubtitle{assessment}
            \begin{itemize}
                \item	\textbf{40\%: assignments} (equally weighted): projected deadlines see syllabus
                        \begin{enumerate}
                            \item	convolution and FIR filters
                            \item   Fourier analysis
                            \item   Dither \& Noise Shaping
                            \item   modulated audio effects
                            \item   compressor \& limiter
                            \item   (phase vocoder)
                        \end{enumerate}

                \smallskip
                \item   \textbf{10\%: mid-term exam I}
                \item   \textbf{10\%: mid-term exam II}

                \smallskip
                \item   \textbf{5\%: participation}

                \smallskip
                \item   \textbf{15\%: quizzes}
                    \begin{itemize}
                        \item   every week?! 
                    \end{itemize}
                
                \smallskip
                \item   \textbf{20\% codec competition}
            \end{itemize}
        \end{frame}

    \section{to do}
        \begin{frame}\frametitle{organizational}\framesubtitle{to do}
            \begin{enumerate}
                %\item   \textbf{find group partner}\\
                %mixed group (1st year + 2nd year) preferred
            
                \smallskip
                \item<2->   \textbf{install Matlab} (Octave/Freemat)
            
                \smallskip
                \item<3->   \textbf{brush up} your math and Matlab
            
            \end{enumerate}
        \end{frame}

\end{document}

