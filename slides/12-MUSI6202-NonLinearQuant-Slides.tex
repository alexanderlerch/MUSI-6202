% move all configuration stuff into one file so we can focus on the content
\input{../shared/common}
\input{../shared/definitions}


\subtitle{Part 12: Non-linear Quantization}

%%%%%%%%%%%%%%%%%%%%%%%%%%%%%%%%%%%%%%%%%%%%%%%%%%%%%%%%%%%%%%%%%%%%%%%%%%%%
\begin{document}
    % generate title page
	\title[]{Digital Signal Processing for Music}   
\author[alexander lerch]{alexander lerch} 
%\institute{~}
%\date[Alexander Lerch]{}
\titlegraphic{\vspace{-16mm}\includegraphics[width=\textwidth,height=3cm]{title}}


\begin{frame}
    \titlepage
    %\vspace{-5mm}
    \begin{flushright}
        \href{http://www.gtcmt.gatech.edu}{\includegraphics[height=.8cm,keepaspectratio]{Logo_GTCMT_black}}
    \end{flushright}
\end{frame}


\section[intro]{introduction}

\begin{frame}{non-linear quantization}{introduction: linear quantization---SNR \& PDF}
		\begin{equation*}
			SNR = 6.02\cdot w + c_{\mathrm{S}}\quad [dB]
		\end{equation*}
		\pause
        \bigskip
		$\Rightarrow c_{\mathrm{S}}$ depends on signal's PDF (and scaling)
		\begin{table}
			\centering
			\begin{footnotesize}
				\begin{tabular}{clc}
				\hline
				\textbf{PDF} & \textbf{SNR}\\
				\hline
				square wave & $c_S =  4.8$\\
				sine wave & $c_S =  1.8$\\
				rect & $c_S =  0$\\
				tri & $c_S \approx  -3$\\
				Gauss & $c_S \approx  -7$\\
				Laplace & $c_S \approx  -9$\\
				speech & $c_S \approx  -10\ldots -15$\\
				\hline
				\end{tabular}
			\end{footnotesize}
		\end{table}
	\end{frame}	
    
	\begin{frame}{non-linear quantization}{introduction}
		\textbf{idea}: quantize frequent signal values at higher resolution
		
		\begin{itemize}
			\item<2-> \textbf{approach 1 }	
				\begin{enumerate}
					\item	flatten PDF (companding)
					\item	linear quantization
					\item	extract signal (expanding)
				\end{enumerate}
			\only<2>{
            \begin{figure}
                \begin{center}
                    \includegraphics[scale=0.5]{Graph/quant_compexp}
                \end{center}
            \end{figure}
            }
			\only<3->{
            \begin{figure}
                \begin{center}
                    \includegraphics[scale=0.2]{Graph/quant_compexp}
                \end{center}
            \end{figure}
            }
            \item<3-> \textbf{approach 2 }	
				\begin{enumerate}
					\item	adapt quantization step size to PDF
				\end{enumerate}
            \bigskip
            \item<4->[$\Rightarrow$] both approaches are equivalent in their result
		\end{itemize}
	\end{frame}	

\section[A-law]{A-Law quantization}
	\begin{frame}{non-linear quantization}{A-Law quantization  (ITU-T G.711) 1/3}
		\begin{equation*}
			F(x)	= sign(x)\left\lbrace
					\begin{array}{ll} 
			          \frac{A|x|}{1+\log(A)}, & |x| \leq \frac{1}{A}\\ 
			          \frac{1+\log(A|x|)}{1+\log(A)}, & \frac{1}{A} \leq |x| \leq 1\\ 
          			\end{array} 
          			\right.
		\end{equation*}
        \bigskip
		\begin{equation*}
			F^{-1}(y)	= sign(y)\left\lbrace
					\begin{array}{ll} 
			          \frac{|y|(1+\log(A))}{A}, & |y| \leq \frac{1}{1+\log(A)}\\ 
			          \frac{\exp\big(|y|(1+\log(A))-1\big)}{A}, & \frac{1}{1+\log(A)} \leq |y| \leq 1\\ 
          			\end{array} 
          			\right.
		\end{equation*}
		\bigskip
		\bigskip
        with $A = 87.7$
        
        \pause
        \begin{itemize}
            \item   linear and high resolution for small amplitudes
            \item   log and increasingly low resolution for high amplitudes
        \end{itemize}
	\end{frame}	

	\begin{frame}{non-linear quantization}{A-Law quantization 2/3}
        \vspace{-4mm}
	    \begin{figure}
			\centering
				\includegraphics[scale=0.5]{Graph/a-law}
		\end{figure}
	\end{frame}

	\begin{frame}{non-linear quantization}{A-Law quantization 3/3}
	    \vspace{-5mm}
        \begin{figure}
			\centering
				\includegraphics[scale=0.5]{Graph/snr_a-law}
		\end{figure}
        \begin{itemize}
            \item   range I: SNR is linear regardless of input level
            \item   range II: SNR increases with input level
        \end{itemize}
	\end{frame}

\section[$\mu$-law]{$\mu$-Law quantization}
	\begin{frame}{non-linear quantization}{$\mu$-Law quantization  (ITU-T G.711)}
		\begin{equation*}
			F(x)	= sign(x)\frac{\log(1+\mu|x|)}{\log(1+\mu)}
		\end{equation*}
        \bigskip
		\begin{equation*}
			F^{-1}(y)	= sign(y)\frac{1}{\mu}\left((1+\mu)^{|y|}-1\right)
		\end{equation*}
		\bigskip
		\bigskip
        with $\mu = 255$
        
        \pause
        compared to A-Law:
        \begin{itemize}
            \item   higher dynamic range
            \item   higher error at small amplitudes
        \end{itemize}
	\end{frame}	

	\section{summary}	
		\begin{frame}{non-linear quantization}{summary}
            \begin{itemize}
                \item \textbf{advantages} of non-linear quantization
                    \begin{itemize}
                        \item   takes advantage of non-uniform distribution of input
                        \item   in line with non-linear loudness perception of the ear
                        \item[$\Rightarrow$]    similar perceptual quality as higher resolution linear quantization
                    \end{itemize}
                \bigskip
                \item    \textbf{disadvantages}
                    \begin{itemize}
                        \item   processing not easily implemented in non-linear amplitude space
                        \item[$\Rightarrow$] only used for transmission   
                    \end{itemize}
            \end{itemize}
		\end{frame}
 

\end{document}

