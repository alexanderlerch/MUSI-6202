\subsection{dynamics processing}
\begin{frame}{dynamics processing}{introduction}
	\begin{itemize}
		\item	\textbf{basic principle}
			\begin{itemize}
				\item	\textit{apply time-variant audio gain} 
                \item   gain depends on signal properties or external factors
			\end{itemize}
		\pause
        \bigskip
		\item	\textbf{applications}
			\begin{itemize}
				\item	avoid clipping (unknown input level)
				\item	suppress noise
				\item	adjust playback level (playlist)
				\item	decrease dynamic range (environmental noise)
				\item	increase loudness/energy (commercials)
				\item	adjust (recording) level
			\end{itemize}
	\end{itemize}
\end{frame}
\begin{frame}{dynamics processing}{introduction: effects}
	\begin{itemize}
		\item	(noise) \textbf{gate}
			\begin{itemize}
				\item	suppression of low levels in pauses
			\end{itemize}
		\pause
        \smallskip
		\item	\textbf{compressor}
			\begin{itemize}
				\item	reduction of the dynamic range
			\end{itemize}
		\pause
        \smallskip
		\item	\textbf{expander}
			\begin{itemize}
				\item	expansion of the dynamic range
			\end{itemize}
		\pause
        \smallskip
		\item	\textbf{limiter}
			\begin{itemize}
				\item	limitation of maximum gain
			\end{itemize}
		\pause
        \smallskip
		\item	\textbf{AGC} (automatic gain control)
			\begin{itemize}
				\item	slow adaptation of recording/payback gain
			\end{itemize}
	\end{itemize}
\end{frame}

\begin{frame}{dynamics processing}{overview}
    \begin{figure}
		\begin{center}
        \begin{picture}(50,25)

            %boxes
            \put(13,0){\framebox(10,10){\tiny{\shortstack[c]{Level\\ Detect.}}}}
            \put(28,0){\framebox(10,10){\tiny{\shortstack[c]{Gain\\ Comp.}}}}

		
            %lines horizontal
            \put(0,20){\vector(1,0){39}}
            \put(41,20){\vector(1,0){10}}

            \put(10,5){\vector(1,0){3}}
            \put(38,5){\line(1,0){2}}
            \put(23,5){\vector(1,0){5}}
            
            %lines vertical
            \put(10,20){\line(0,-1){15}}
            
            \put(40,5){\vector(0,1){14}}
            
            %circles
            \put(38.5,19){$\otimes$} %40,20
            
            \put(10,20){\circle*{1}}

            %text
            \put(0,21){\footnotesize{\shortstack[c]{$x(i)$}}}
            \put(48,21){\footnotesize{\shortstack[c]{$y(i)$}}}
            \put(42,17){\footnotesize{\shortstack[c]{$g(i)$}}}

        \end{picture}
		\end{center}
    \end{figure}
	\begin{equation*}
		y(i) = x(i) \cdot g\textcolor{gtgold}{(i)}
	\end{equation*}	
	\pause
	
	computation of $g(i)$ usually depends on
	\begin{enumerate}
		\item	input signal \textit{level}
		\item	properties \& characteristics of the dynamics processor
		\item	time-based control mechanism
	\end{enumerate}
\end{frame}

\begin{frame}{dynamics processing}{level detection}
	\vspace{-5mm}
    \begin{itemize}
        \item typical measures
    \begin{itemize}
		\item	\textbf{peak}:\\ physical measure of maximum amplitude
		\item	\textbf{rms}:\\ physical measure of power level
		\item	\textbf{loudness model}:\\ models of loudness perception (dBA, Zwicker, BS.1770)
		\vspace{5mm}
    \end{itemize}
    \pause
    \item   \textbf{level computation}

		\begin{equation*}
			v_\mathrm{dB}(i) = 20\cdot\log_{10}\left(\frac{v(i)}{v_0}\right)
		\end{equation*}
	
		\pause
		\begin{itemize}
			\item	$v_0$: reference constant (\unit[0]{dB} point)
			
					digital: $v_0 = 1$ $\Rightarrow \unit{dBFS}$
			\item	scaling	factor: \unit[1]{dB} $\approx$ JNDL
		\end{itemize}

	\end{itemize}
\end{frame}
%\begin{frame}{dynamics processing}{level detection: human intensity perception}
	%decibel scale is \textit{not} loudness scale:
	%\begin{itemize}
		%\item	equal-sized steps on the decibel scale not perceived as equal-sized loudness steps
	%\end{itemize}
	%
	%\pause
	%\bigskip
	%perceptual loudness depends on
	%\begin{itemize}
		%\item	frequency
		%\item	cochlear resolution
		%\item	masking effects
	%\end{itemize}
	%\vspace{-20mm}
	%\begin{figure}
		%\flushright
			%\includegraphics[scale=.25]{graph/equalloudness}
	%\end{figure}
%\end{frame}
\begin{frame}{dynamics processing}{level detection: root mean square 1/2}
	\begin{equation*}\label{eq:rms}
		\input{\AcaEq/Llf_RMS}
	\end{equation*}
	\begin{figure}
		\includegraphics[scale=.7]{\AcaGraph/rms}
	\end{figure}
\end{frame}
\begin{frame}{dynamics processing}{level detection: root mean square 2/2}
			%\begin{itemize}
				%\item	measure of power
				%\item	integration time:
				%\begin{footnotesize}
					%\begin{equation*}
						%T_\mathrm{I} = \frac{i_\mathrm{e}(n)-i_\mathrm{s}(n)}{f_\mathrm{s}}
					%\end{equation*}
				%\end{footnotesize}
			%\end{itemize}
			%
			%\pause
			\textbf{sample-by-sample processing}:
			\begin{itemize}
				\item	reduce computational complexity
					\begin{footnotesize}
					\begin{eqnarray*}
						v^2_{\mathrm{RMS}}(n) &=& \frac{x(i_{\mathrm{e}}(n))^2 - x(i_{\mathrm{s}}(n-1))^2}{i_{\mathrm{e}}(n)-i_{\mathrm{s}}(n) + 1} + v^2_{\mathrm{RMS}}(n-1) \\
						v_{\mathrm{RMS}}(n)	&=& \sqrt{v^2_{\mathrm{RMS}}(n)}
					\end{eqnarray*}
					\end{footnotesize}
				\pause
				\item	single pole approximation
					\begin{footnotesize}
					\begin{eqnarray*}
						v_\mathrm{tmp}(i)	&=& \alpha\cdot v_\mathrm{tmp}(i-1) + (1-\alpha)\cdot x(i)^2\\
						v^*_{\mathrm{RMS}}(i)		&=& \sqrt{v_\mathrm{tmp}(i)}
					\end{eqnarray*}
					\end{footnotesize}
			\end{itemize}
\end{frame}
\begin{frame}{dynamics processing}{level detection: weighted root mean square}
		\vspace{-5mm}
		\begin{figure}
			\centering
			\input{\AcaPict/lowlevelfeatures_rmsblock}
	    \end{figure}
	    
		\pause
        \vspace{-10mm}
	    $H(z)$:
	    \begin{itemize}
	    	\item	A, B, C weighting
	    	\item	RLB (BS.1770)
	    	\item	\ldots
	    \end{itemize}
	    
		\uncover<2->{
        \begin{figure}
			\centering
			\includegraphics[scale=.7]{\AcaGraph/loudnessweighting}
			\label{fig:loudnessweighting}
		\end{figure}
        }
\end{frame}
\begin{frame}{dynamics processing}{level detection: peak detection (PPM) 1/2}
		\begin{figure}
			\centering
			\input{\AcaPict/lowlevelfeatures_ppm}
		\end{figure}	
		\begin{itemize}
			\only<1>{
			\item[]
			\item[]
			\item[]
			\item[]
			}
			\only<2-3>{
			\item \textbf{release state} ($|x(i)| \leq v_{\mathrm{PPM}}(i-1)\Rightarrow\lambda = \alpha_\mathrm{RT}$)
				\invisible<2>{
				\begin{eqnarray*}
					v_{\mathrm{PPM}}(i) &=& v_{\mathrm{PPM}}(i-1) - \alpha_\mathrm{RT}\cdot v_{\mathrm{PPM}}(i-1)\nonumber\\
					\pause
								&=& (1-\alpha_\mathrm{RT})\cdot v_{\mathrm{PPM}}(i-1) 
				\end{eqnarray*}
				}
			}
			\only<4-5>{
			\item \textbf{attack state} ($|x(i)| > v_{\mathrm{PPM}}(i-1)\Rightarrow\lambda = 0$)
				\invisible<4>{
				\begin{eqnarray*}
					v_{\mathrm{PPM}}(i) &=& \alpha_\mathrm{AT}\cdot\big(|x(i)| - v_{\mathrm{PPM}}(i-1)\big) + v_{\mathrm{PPM}}(i-1)\nonumber\\
					\pause
								&=& \alpha_\mathrm{AT}\cdot |x(i)| + (1-\alpha_\mathrm{AT})\cdot v_{\mathrm{PPM}}(i-1) 
				\end{eqnarray*}
				}
			}
		\end{itemize}
\end{frame}
\begin{frame}{dynamics processing}{level detection: peak detection (PPM) 2/2}
		\begin{figure}
			\centering
				\includegraphics[scale=.7]{\AcaGraph/ppm}
			\label{fig:ppm_level}
		\end{figure}
\end{frame}	

\begin{frame}{dynamics processing}{response curve: limiter}
	\only<1>{
	\begin{figure}
		\centering
			\includegraphics[scale=.6]{graph/fx3_lmt-01}
	\end{figure}
	}
	\only<2>{
		\begin{figure}
			\centering
				\includegraphics[scale=.7]{graph/dynamicsresponsecurve_1}
		\end{figure}

		param $LT=\unit[-9]{dB}$ \includeaudio{audio/svdynamicsresponsecurve_1.mp3}
	}
\end{frame}
\begin{frame}{dynamics processing}{response curve: compressor}
	\only<1>{
	\begin{figure}
		\centering
			\includegraphics[scale=.6]{graph/fx3_comp-01}
	\end{figure}
	}
	\only<2>{
		\begin{figure}
			\centering
				\includegraphics[scale=.7]{graph/dynamicsresponsecurve_2}
		\end{figure}

		param $CT=\unit[-9]{dB}$\includeaudio{audio/svdynamicsresponsecurve_2.mp3}
	}
\end{frame}
\begin{frame}{dynamics processing}{response curve: expander}
	\only<1>{
	\begin{figure}
		\centering
			\includegraphics[scale=.6]{graph/fx3_exp-01}
	\end{figure}
	}
	\only<2>{
		\begin{figure}
			\centering
				\includegraphics[scale=.7]{graph/dynamicsresponsecurve_3}
		\end{figure}

		param $ET=\unit[-6]{dB}$\includeaudio{audio/svdynamicsresponsecurve_3.mp3}
	}
\end{frame}
\begin{frame}{dynamics processing}{response curve: noise gate}
	\only<1>{
	\begin{figure}
		\centering
			\includegraphics[scale=.6]{graph/fx3_gate-01}
	\end{figure}
	}
	\only<2>{
		\begin{figure}
			\centering
				\includegraphics[scale=.7]{graph/dynamicsresponsecurve_4}
		\end{figure}

		param $NT=\unit[-12]{dB}$\includeaudio{audio/svdynamicsresponsecurve_4.mp3}
	}
\end{frame}

\begin{frame}{dynamics processing}{response curve: mathematical description (compressor)}
	\vspace{-5mm}
    logarithmic description, nonlinear part
    \bigskip
	\begin{itemize}
		\item	 \textbf{output}:
			$Y = g(X) + X\quad [\unit{dB}]$
		\pause
		\item	\textbf{ratio}:
			$R = \frac{\Delta L_i}{\Delta L_o}$
		\pause
		\item	\textbf{slope}:
			$CS = 1-\frac{1}{R}$
		\pause
		\item	\textbf{linear equation} (offset CT):
			$Y = \frac{1}{R}\left(X-CT\right) + CT$
		\pause
		\item	\textbf{gain} ($g = Y-X$):
			\begin{footnotesize}\begin{eqnarray*}
				g &=& \frac{1}{R}(X-CT) + CT - X\\
				\pause
				  &=& \left(1-\frac{1}{R}\right)\cdot (CT-X)\\
                  &=& CS\cdot(CT-X)
			\end{eqnarray*}\end{footnotesize}
	\end{itemize}
\end{frame}
\begin{frame}{dynamics processing}{response curve: mathematical description (summary 1/2)}
	\vspace{-5mm}
    logarithmic description, nonlinear part
    \bigskip
	\begin{itemize}
		\item	\textbf{limiter}
			\begin{footnotesize}\begin{eqnarray*}
				R &=& \infty\\
				Y &=& LT\\
				g &=& LT-X
			\end{eqnarray*}\end{footnotesize}
		\pause
		\item	\textbf{compressor}
			\begin{footnotesize}\begin{eqnarray*}
				R &>& 1\\
				Y &=& \frac{1}{R}\left(X-CT\right) + CT\\
				g &=& \left(1-\frac{1}{R}\right)\cdot (CT-X)
			\end{eqnarray*}\end{footnotesize}
	\end{itemize}
\end{frame}
\begin{frame}{dynamics processing}{response curve: mathematical description (summary 2/2)}
	\vspace{-5mm}
    logarithmic description, nonlinear part
    \bigskip
	\begin{itemize}
		\item	\textbf{expander}
			\begin{footnotesize}\begin{eqnarray*}
				R &<& 1\\
				Y &=& \frac{1}{R}\left(X-ET\right) + ET\\
				g &=& \left(1-\frac{1}{R}\right)\cdot (ET-X)
			\end{eqnarray*}\end{footnotesize}
		\pause
		\item	\textbf{gate}
			\begin{footnotesize}\begin{eqnarray*}
				R &=& 0\\
				Y &=& -\infty\\
				g &=& -\infty
			\end{eqnarray*}\end{footnotesize}
	\end{itemize}
\end{frame}

\begin{frame}{dynamics processing}{smoothing: attack and release 1/2}
	\begin{figure}
		\centering
			\includegraphics[scale=.5]{graph/smooth_control}
	\end{figure}
	\begin{itemize}
		\item	$\alpha_{AT}$: attack constant
		\item	$\alpha_{RT}$: release constant
	\end{itemize}
	
	\pause
	\begin{eqnarray*}
		g(n) &=& \alpha\cdot(f(n)-g(n-1)) + g(n-1)\nonumber\\
		\pause
			&=& \alpha f(n) + (1-\alpha)\cdot g(n-1)
	\end{eqnarray*}
\end{frame}

\begin{frame}{dynamics processing}{smoothing: attack and release 2/2}
	\vspace{-5mm}
    \begin{figure}
		\centering
			\includegraphics[scale=1]{graph/comp_sine}
	\end{figure}
\end{frame}

\begin{frame}{dynamics processing}{smoothing: attack and release coefficients}
    \begin{itemize}
        \item   single pole step response $\rightarrow g(t) = 1-e^{\frac{-t}{\tau}}$ 
        \pause
        \item   define single pole integration time between 10\% and 90\%
            \begin{eqnarray*}
                t_\mathrm{I} &=& t_{90} - t_{10}\\
                0.1 &=& 1 - e^{\frac{-t_{10}}{\tau}}\\
                0.9 &=& 1 - e^{\frac{-t_{90}}{\tau}}\\
        \pause
            \Rightarrow \nicefrac{0.9}{0.1} &=& e^{\frac{t_{90}-t_{10}}{\tau}}\\
        \pause
            \log\left(\nicefrac{0.9}{0.1}\right) &=& \nicefrac{t_{90}-t_{10}}{\tau}\\
        \pause
            t_{90}-t_{10} &=& 2.197\tau\\
            \tau &\approx& \nicefrac{t_\mathrm{I}}{2.2}
        \end{eqnarray*}

    \end{itemize}
\end{frame}

\begin{frame}{dynamics processing}{overall system: limiter}
	\begin{figure}
		\centering
			\includegraphics[scale=.5]{graph/limiter}
	\end{figure}
	\begin{equation*}
		CS = 1-\frac{1}{R} \Rightarrow LS = 1
	\end{equation*}

	\pause	
	\begin{itemize}
		\item	$X<LT\rightarrow g = 1$
		\pause
		\item	$X>LT\rightarrow g = (LT-X)$
	\end{itemize}
\end{frame}

\begin{frame}{dynamics processing}{gain visualization: combined system}
	\vspace{-5mm}\begin{figure}
		\centering
			\includegraphics[scale=.25]{graph/dynamics_gain}
	\end{figure}
\end{frame}

\begin{frame}{dynamics processing}{audio examples}
    \includeaudio{audio/sv.mp3}
    \bigskip
    \begin{itemize}
        \item Gate \includeaudio{audio/sv_Gate.mp3}
        \item Expander \includeaudio{audio/sv_Expander.mp3}
        \item Compressor \includeaudio{audio/sv_Compressor.mp3}
        \item Limiter \includeaudio{audio/sv_Limiter.mp3}
    \end{itemize}
\end{frame}

\begin{frame}{dynamics processing}{variants 1/3}
	\vspace{-5mm}
    \begin{itemize}
		\item	\textbf{attack \& release constant selection}
			\begin{itemize}
				\item	depending on ``abruptness'' of change
			\end{itemize}
		\pause
        \smallskip
		\item	\textbf{hold time}
			\begin{itemize}
				\item	before release, hold gain constant (avoid pumping with low frequency signals)
			\end{itemize}
		\pause
        \smallskip
		\item	\textbf{oversampling}
			\begin{itemize}
				\item	high time resolution for peak detection
			\end{itemize}
	\end{itemize}
    \visible<3->{
    \begin{figure}[!htbp]
        \centering
            \includegraphics[scale=.6]{graph/hiddenclipping}
    \end{figure}
    }
\end{frame}

\begin{frame}{dynamics processing}{variants 2/3}
	\vspace{-5mm}
    \begin{itemize}
		\item	\textbf{stereo link}
			\begin{itemize}
				\item	consider both channels (avoid level-dependent changes of stereo image)
					\pause
					\begin{itemize}
						\item	one master channel (left or right)
						\item	mean of both channels
						\item	channel with higher level (max)
					\end{itemize}
			\end{itemize}
		\pause
        \smallskip
        \smallskip
		\item	\textbf{soft knee}
			\begin{itemize}
				\item	smooth crossover from linear area to compressed area
					\visible<4->{
                    \begin{figure}
						\flushright
							\includegraphics[scale=.2]{graph/dynamics_softknee}
					\end{figure}
                    }
					\vspace{-20mm}
					\pause
					potentially noticeable with
					\begin{itemize}
						\item	very short attack times
						\item	high compression ratios	
					\end{itemize}
			\end{itemize}
	\end{itemize}
\end{frame}

\begin{frame}{dynamics processing}{variants 3/3}
	\begin{itemize}
		\item	\textbf{side chain}
			\begin{itemize}
				\item	choose different input signal for level control (``ducking'')
			\end{itemize}
		\pause
        \bigskip
		\item	\textbf{look-ahead}
			\begin{itemize}
				\item	introduce higher delay in signal path
					\begin{itemize}
						\item	shift gain modification in time
						\item	combine ``future'' measurement with current
					\end{itemize}
			\end{itemize}
        \bigskip
		\pause
		\item	\textbf{multi-band compression}
            \begin{itemize}
                \item	apply one compressor to each frequency band
                \item	advantages:
                    \begin{itemize}
                        \item	avoid pumping: varying level in one band (e.g. bass drum) does not influence gain of other bands
                        \item	maximize power, overall loudness
                    \end{itemize}
            \end{itemize}
        
	\end{itemize}
\end{frame}

\begin{frame}{dynamics processing}{parameter ranges}
	\vspace{-5mm}
    \begin{itemize}
		\item	\textbf{threshold}\\
		\uncover<2->{$-120 \ldots 0$ dB}
		
		\item	\textbf{ratio}\\
		\uncover<2->{$0.05 \ldots 20$ (Limiter: $\infty$)}
		
		\item	\textbf{attack}\\
		\uncover<2->{$0 \ldots 10$ \unit{ms}}
		
		\item	\textbf{release}\\
		\uncover<2->{$20 \ldots 300$ \unit{ms}}

		\item	\textbf{hold}\\
		\uncover<2->{$0 \ldots 10$ \unit{ms}}
		
		\item	\textbf{stereo-link}\\
		\uncover<2->{On/Off}
		
		\item	\textbf{oversampling}\\
		\uncover<2->{$1 \ldots 8$}
		
		\item	\textbf{look-ahead}\\
		\uncover<2->{$0 \ldots 500$ \unit{ms}}
	\end{itemize}
\end{frame}

\begin{frame}{dynamics processing}{summary}
	dynamics processing systems are
	\begin{itemize}
		\item	\textbf{time variant}:\\ gain changes over time
		\pause
		\item	\textbf{signal adaptive}:\\ gain depends on (input) signal
		\pause
		\item	sometimes \textbf{non-linear}:\\ at very short attack times (limiting)
	\end{itemize}
\end{frame}

